\chapter[Análisis de requisitos]{Análisis de requisitos}
\label{Chap2}

Antes de abordar el desarrollo y la estructura técnica de la aplicación, resulta fundamental realizar un análisis de los requisitos funcionales y no funcionales que definirán el alcance del Ecosistema de Aprendizaje. Este análisis tiene como objetivo identificar las necesidades específicas de la organización, los usuarios involucrados y las funcionalidades esenciales que el sistema deberá cubrir para cumplir con los objetivos planteados.

\section{Requisitos funcionales}

Los requisitos funcionales describen las acciones y procesos que la aplicación deberá permitir realizar a sus usuarios. A partir de las reuniones iniciales con Tracasa Global y Tracasa Instrumental, y del estudio de los flujos actuales de gestión de la formación, se han definido las siguientes funcionalidades principales:
\begin{description}
	\item[Gestión de acciones formativas:]  permitir la creación, edición, eliminación y clonación de acciones formativas, distinguiendo entre formaciones planificadas y extraordinarias.
	
	\item[Gestión de solicitudes:] permitir que los empleados soliciten nuevas formaciones o su participación en acciones ya existentes, con validación por parte de responsables, directores y el área de PyT (Proyectos y Tecnología).
	
	\item[Flujo de aprobación y notificaciones:] automatizar el envío de correos electrónicos para informar a los distintos roles del estado de las solicitudes, aprobaciones y valoraciones pendientes.
	
	\item[Gestión de docentes y asistentes:] permitir asociar docentes internos o externos a una formación, así como gestionar los asistentes y sus convocatorias.
	
	\item[Encuestas y valoraciones:] generar encuestas de eficacia formativa y registrar valoraciones asociadas a cada acción formativa.
	
	\item[Planes de formación anual:] posibilitar la creación, edición y seguimiento de planes de formación, con la asignación de propuestas por departamento y la gestión de permisos para cada usuario.
	
	\item[Integración con plataformas externas:] permitir el acceso directo a la plataforma de aprendizaje GoodHabitz y a la plataforma de ciberseguridad corporativa desde la propia aplicación.
	
	\item[Histórico y trazabilidad:] registrar los cambios y actualizaciones realizados sobre las acciones formativas para mantener un seguimiento completo de la evolución de cada proceso.
	
	
\end{description}

\section{Requisitos no funcionales}

Además de las funcionalidades descritas, la aplicación deberá cumplir con una serie de condiciones relacionadas con su rendimiento, seguridad y usabilidad:

\begin{description}
	\item[Usabilidad:] la interfaz debe ser intuitiva, clara y accesible para usuarios con distintos niveles de experiencia técnica.
	
	\item[Seguridad:] el acceso a la aplicación deberá estar autenticado y controlado mediante un sistema de roles (solicitante, responsable, director, PyT o administrador), garantizando la protección de los datos personales y formativos.
	
	\item[Rendimiento:] el sistema debe ser capaz de gestionar múltiples solicitudes y consultas simultáneas sin afectar la velocidad de respuesta.
	
	\item[Escalabilidad:] la arquitectura debe permitir la incorporación de nuevas funcionalidades o módulos sin afectar el funcionamiento existente.
	
	\item[Mantenibilidad:] el código debe estar bien estructurado y documentado, siguiendo principios de modularidad y reutilización.
	
	\item[Compatibilidad:] la aplicación debe ser accesible desde los navegadores corporativos más utilizados y adaptarse correctamente a diferentes resoluciones de pantalla.
\end{description}

\section{Roles}
Durante la fase de análisis se identificaron los diferentes tipos de usuarios que interactuarán con el sistema, cada uno con permisos y responsabilidades específicas:

\begin{description}
	\item[Solicitante:] puede solicitar nuevas acciones formativas o inscribirse en formaciones ya creadas.
	
	\item[Responsable de área:] recibe y valida las solicitudes de los empleados de su equipo.
	
	\item[Director de departamento:] revisa y aprueba las solicitudes validadas por los responsables de las áreas que forman parte de su departamento.
	
	\item[PyT/PyV (Proyectos y Tecnología):] gestiona las solicitudes aprobadas, tramita las formaciones y supervisa el cumplimiento de los procesos.
	
	\item[Administrador del sistema:] mantiene la aplicación, gestiona usuarios y configura los parámetros globales de funcionamiento.
\end{description}