\chapter[Tecnologías y herramientas utilizadas]{Tecnologías y herramientas utilizadas}
\label{Chap3}

La elección de las tecnologías empleadas en el desarrollo del Ecosistema de Aprendizaje responde a la necesidad de construir una solución robusta, segura y con capacidad de crecer a largo plazo dentro del entorno corporativo de Tracasa Global y Tracasa Instrumental. En este sentido, se van a seleccionado herramientas consolidadas en el sector y ampliamente utilizadas en el desarrollo de aplicaciones empresariales, garantizando así un mantenimiento eficiente, una curva de aprendizaje asumible y una larga vida útil del sistema.

\section{Lenguaje de programación y ecosistema .NET}

Antes de profundizar en las tecnologías utilizadas en el desarrollo del backend, resulta
pertinente introducir brevemente el concepto de \textbf{framework}, ya que constituye la base
sobre la que se construye gran parte de la solución implementada.

Un \textit{framework} \cite{framework} puede entenderse como un conjunto de herramientas, componentes y
estructuras previamente definidas que facilitan la creación de aplicaciones de manera más
ágil y eficiente. Su propósito es evitar que los desarrolladores deban partir desde cero en cada proyecto, proporcionando elementos reutilizables y mecanismos estandarizados para resolver problemas comunes. Gracias a ello, se reduce el tiempo de desarrollo, se mejora la calidad del software y se garantiza una mayor consistencia en la implementación.

En este contexto, la plataforma \textbf{.NET} constituye uno de los frameworks más completos y ampliamente adoptados en el ámbito empresarial. Se trata de una plataforma de código abierto y multiplataforma diseñada para compilar y ejecutar aplicaciones de todo tipo, desde soluciones web hasta servicios distribuidos o herramientas de escritorio. .NET destaca por su alto rendimiento, su sistema avanzado de ejecución y su amplio conjunto de bibliotecas optimizadas, que permiten trabajar de forma segura, productiva y escalable. Entre sus características clave se encuentran la administración automática de memoria, la seguridad de tipos, la protección de la memoria y la existencia de un extenso ecosistema de herramientas y marcos de trabajo especializados.

Sobre esta plataforma se apoya el lenguaje \textbf{C\#}, utilizado como lenguaje principal en el desarrollo del Ecosistema de Aprendizaje. C\# es un lenguaje moderno, fuertemente tipado y orientado a objetos, diseñado para ofrecer productividad y claridad en el código, además de funciones avanzadas como la concurrencia integrada, el manejo automático de recursos y una sintaxis elegante fácil de mantener a largo plazo. Su interoperabilidad con otros lenguajes del ecosistema .NET y su alta fiabilidad lo han convertido en la opción predilecta dentro del entorno profesional cuando se requieren aplicaciones robustas, seguras y de gran rendimiento.

La elección conjunta de C\# y .NET proporciona una base tecnológica sólida para el proyecto, especialmente en un entorno corporativo donde la seguridad, la escalabilidad y la mantenibilidad son requisitos fundamentales.

Entre los motivos que justifican el uso de esta tecnología destacan los siguientes:

\begin{itemize}
	\item \textbf{Madurez y estabilidad}: .NET es una plataforma ampliamente consolidada y en constante evolución, lo que garantiza fiabilidad incluso en proyectos de larga
	duración.
	
	\item \textbf{Rendimiento elevado}: el entorno de ejecución de .NET ofrece tiempos de
	respuesta competitivos, ideales para sistemas corporativos que gestionan un gran
	volumen de solicitudes simultáneas.
	
	\item \textbf{Seguridad integrada}: la plataforma incorpora mecanismos nativos de protección,	validación y control de acceso, esenciales en aplicaciones que manejan datos sensibles como los relativos a formación y gestión interna.
	
	\item \textbf{Productividad y facilidad de mantenimiento}: gracias al uso de C\# y al acceso a bibliotecas optimizadas, el desarrollo es más ágil y el código más fácil de mantener.
	
	\item \textbf{Interoperabilidad y ecosistema amplio}: .NET permite integrar la solución con otros servicios y tecnologías corporativas, reduciendo riesgos y facilitando futuras ampliaciones.
\end{itemize}

En conjunto, el uso del ecosistema .NET y del lenguaje C\# constituye una opción altamente adecuada para el desarrollo del Ecosistema de Aprendizaje, proporcionando un equilibrio óptimo entre rendimiento, seguridad, mantenibilidad y capacidad de evolución tecnológica.


\section{Interfaz de usuario}

La capa de presentación del Ecosistema de Aprendizaje se ha desarrollado combinando
\textbf{TypeScript}, el motor de componentes \textbf{Blazor} y estilos definidos mediante
\textbf{CSS global}. Este conjunto tecnológico permite construir interfaces modernas,
dinámicas y fáciles de mantener, aprovechando tanto el ecosistema web tradicional como
las capacidades integradas de .NET para el desarrollo de aplicaciones interactivas.

\begin{itemize}
	\item \textbf{TypeScript} aporta tipado estático, una organización más clara del código y la detección temprana de errores durante el desarrollo. Esto reduce fallos en ejecución y mejora significativamente la mantenibilidad de la interfaz.
	
	\item \textbf{Blazor} facilita la creación de componentes interactivos empleando código familiar para el equipo de desarrollo, permitiendo integrar lógica de presentación de forma estructurada sin necesidad de depender exclusivamente de JavaScript. Su capacidad para trabajar con datos y eventos de manera fluida mejora la consistencia de la experiencia de usuario.
	
	\item \textbf{CSS} se utiliza para definir estilos unificados en toda la aplicación, garantizando coherencia visual, facilidad de personalización y una apariencia profesional adaptada al entorno corporativo.
\end{itemize}

Este conjunto de tecnologías permite construir una interfaz intuitiva, ligera y adaptada a los estándares actuales de usabilidad, asegurando que la experiencia de usuario resulte
coherente para todos los perfiles que interactúan con la aplicación.

\section{Gestión de datos: Entity Framework y SQL Server}
El acceso a datos se ha implementado mediante \textbf{Entity Framework}, un \textit{ORM} que simplifica la comunicación con la base de datos. Sus principales ventajas incluyen:

\begin{itemize}
	\item \textbf{Reducción del código repetitivo}: permite abstraer consultas comunes sin
	necesidad de escribir SQL manual, disminuyendo la probabilidad de errores.
	
	\item \textbf{Sincronización continua del modelo de datos}: gracias al sistema de migraciones,
	es posible actualizar la base de datos de forma segura a medida que evoluciona el
	proyecto.
	
	\item \textbf{Facilidad de mantenimiento}: su integración natural con C\# permite trabajar de
	forma coherente con las entidades del sistema.
\end{itemize}

Como motor de almacenamiento se ha utilizado \textbf{SQL Server} \cite{sqlServer}, una tecnología ampliamente extendida en entornos corporativos y que presenta ventajas significativas:

\begin{itemize}
	\item \textbf{Potentes capacidades de administración}: SQL Server facilita la gestión de volúmenes grandes de información, el control de transacciones y la seguridad.
	
	\item \textbf{Escalabilidad}: permite afrontar un crecimiento progresivo del sistema sin necesidad de realizar cambios disruptivos en la infraestructura.
	
	\item \textbf{Integración con herramientas empresariales existentes}: lo que agiliza los	procesos de despliegue y mantenimiento.
\end{itemize}

Estas características hacen que la combinación de Entity Framework con SQL Server sea
particularmente adecuada para aplicaciones internas de gestión, donde la consistencia,
seguridad y trazabilidad de los datos son elementos clave.

\section{Sistema de notificaciones y comunicaciones}
La plataforma incorpora un sistema automatizado de \textbf{envío de correos electrónicos},
fundamental para coordinar las aprobaciones, convocatorias, valoraciones y avisos entre
usuarios. Su importancia radica en que:

\begin{itemize}
	\item Reduce los tiempos de gestión.
	\item Evita pérdidas de información entre departamentos.
	\item Asegura una comunicación constante y centralizada.
\end{itemize}

La integración del sistema de notificaciones dentro de la lógica de la aplicación permite que el usuario reciba información en tiempo real, lo que incrementa la eficiencia de los flujos de trabajo relacionados con la formación.

En conjunto, todas estas tecnologías permiten construir una aplicación robusta, segura,
fácil de mantener y preparada para evolucionar en el futuro. Su uso está alineado con las
tendencias actuales en desarrollo de software corporativo, ofreciendo garantías de fiabilidad y sostenibilidad a largo plazo dentro del entorno empresarial de Tracasa.
