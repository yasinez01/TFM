\chapter[Introducción]{Introducción}
\label{Chap1}

 En la actualidad, las organizaciones se encuentran inmersas en un entorno empresarial en constante transformación, donde la gestión del conocimiento y el desarrollo de competencias profesionales se han convertido en factores estratégicos para garantizar la competitividad y la sostenibilidad a largo plazo. La capacidad de formar a los equipos de manera ágil, estructurada y alineada con los objetivos corporativos es un desafío clave para cualquier empresa que aspire a mantener una posición destacada en su sector.

En este contexto, surge el presente Trabajo de Fin de Máster, que tiene como propósito el diseño, desarrollo e implementación de la aplicación web Ecosistema de Aprendizaje en colaboración con Tracasa Global \cite{tracasaGlobal} y Tracasa Instrumental \cite{tracasaInstrumental}. Esta solución tecnológica se concibe como una plataforma integral destinada a optimizar los procesos de gestión del aprendizaje, permitiendo a los trabajadores acceder a un entorno digital centralizado desde el cual pueden solicitar nuevas acciones formativas, inscribirse en programas ya existentes y acceder a diferentes recursos de capacitación externos. 

La aplicación no solo buscará facilitar el acceso a la formación, sino también incorporar un sistema de validación y aprobación por parte de responsables y directores, garantizando así la coherencia entre las necesidades individuales de los empleados y las prioridades estratégicas de la empresa. Además, se prevé que incluya funcionalidades avanzadas como la planificación de actividades, el seguimiento del progreso y la valoración de la eficacia formativa, con el fin de contribuir a una mayor trazabilidad y transparencia en los procesos de desarrollo profesional.

El desarrollo del proyecto se estructurará en distintas fases claramente definidas. En primer lugar, se llevará a cabo una fase de análisis de requisitos, en la que se recopilarán las necesidades funcionales y técnicas de la empresa, identificando los procesos clave a digitalizar y los distintos perfiles de usuario que interactuarán con la aplicación (empleados, responsables, directores y administradores del sistema). A partir de esta información se definirán los requisitos funcionales principales, como la creación y gestión de acciones formativas, la validación de solicitudes, la integración con plataformas externas de aprendizaje, la generación de informes de seguimiento y la automatización de notificaciones por correo. En el Apéndice \ref{anexo} se encuentras figuras de la configuración del correo y las plantillas de correos para personalizar (Figuras \ref{fig:ej36}, \ref{fig:ej37}, \ref{fig:ej38}, \ref{fig:ej39}, \ref{fig:ej40} y \ref{fig:ej41}).

Posteriormente, se realizará la fase de diseño, en la que se definirá la arquitectura de la solución siguiendo un modelo por capas, garantizando la separación de responsabilidades y la escalabilidad del sistema. En esta etapa se elaborarán los diagramas de flujo, los modelos de datos y la estructura de componentes que servirán como base para la fase de implementación.

Durante la fase de desarrollo, se procederá a construir los distintos módulos de la aplicación, que se verá en la sección de \textit{Desarrollo del Ecosistema de Aprendizaje}, utilizando tecnologías modernas del ecosistema \textit{.NET} \cite{netIntroduction} y \textit{C\#} \cite{cSharp}, junto con \textit{TypeScript} \cite{typescript} y \textit{HTML\cite{html} / CSS\cite{css}} para la capa de presentación. Se aplicará el patrón arquitectónico \textit{Command Query Responsibility Segregation (CQRS)} \cite{cqrs} para separar las operaciones de lectura y escritura, asegurando así un código más limpio y mantenible.

Una vez implementada la base del sistema, se llevará a cabo la fase de pruebas, en la que se realizarán tanto pruebas unitarias como de integración \cite{integracion}, además de validaciones funcionales con usuarios de la empresa colaboradora. Este proceso permitirá detectar posibles errores, ajustar la experiencia de usuario y garantizar la fiabilidad del sistema antes de su despliegue.

Finalmente, la fase de despliegue y entrega contemplará la puesta en marcha de la aplicación en el entorno corporativo, la configuración de las bases de datos y servidores, así como la formación de los usuarios en el uso de la herramienta.

Se estima que el desarrollo completo del proyecto se extenderá a lo largo de aproximadamente 1 año, combinando sesiones de planificación y seguimiento con el equipo técnico de Tracasa y el área de gestión de proyectos tecnológicos (PyT). Para ello, se adoptará una metodología ágil de trabajo, que permitirá iterar de manera continua sobre los avances, incorporar mejoras basadas en el feedback de los usuarios y priorizar las funcionalidades más relevantes para cada perfil.

En definitiva, este proyecto representa una apuesta firme por la innovación y la transformación digital en el ámbito de la formación corporativa. La implementación del Ecosistema de Aprendizaje permitirá a Tracasa Global y Tracasa Instrumental fortalecer su modelo de gestión del talento, fomentar el aprendizaje continuo y consolidar una cultura organizativa orientada al crecimiento y a la excelencia.

\section{Contribución a los Objetivos de Desarrollo Sostenible (ODS)}
El desarrollo de soluciones tecnológicas orientadas a la mejora de los procesos internos de las organizaciones tiene un impacto que trasciende el ámbito puramente técnico. En este sentido, el Ecosistema de Aprendizaje diseñado en este Trabajo de Fin de Máster no solo responde a necesidades operativas de gestión de la formación, sino que también contribuye a impulsar prácticas alineadas con los principios de sostenibilidad, innovación y desarrollo humano. La implementación de una plataforma digital que fomenta el aprendizaje continuo y la mejora profesional permite establecer una relación directa con varios de los Objetivos de Desarrollo Sostenible definidos a nivel internacional.

\begin{itemize}
	\item \textbf{ODS 4: Educación de calidad}: El proyecto promueve una educación de calidad al facilitar el acceso a la formación continua dentro del entorno corporativo. La plataforma desarrollada centraliza y organiza las acciones formativas, permitiendo a los empleados acceder a recursos de aprendizaje, realizar solicitudes de formación y evaluar su eficacia. De este modo, se fomenta el aprendizaje permanente, la actualización de competencias profesionales y la mejora de la cualificación del personal, contribuyendo a una formación más estructurada, accesible y alineada con las necesidades reales de la organización.
	
	\item \textbf{ODS 8: Trabajo decente y crecimiento económico}: El Ecosistema de Aprendizaje contribuye a la mejora de las condiciones laborales al impulsar el desarrollo profesional y la empleabilidad de los trabajadores. Una gestión eficiente de la formación favorece la adquisición de nuevas competencias, incrementa la productividad y refuerza el crecimiento sostenible de la organización. Asimismo, el sistema promueve la igualdad de oportunidades en el acceso a la formación, apoyando un entorno laboral más justo, motivador y orientado al talento.
	
	\item \textbf{ODS 9: Industria, innovación e infraestructura}:
	El desarrollo de una aplicación web para la gestión integral de la formación corporativa representa una apuesta por la innovación tecnológica y la digitalización de procesos internos. La solución propuesta mejora la infraestructura digital de la organización, optimiza los flujos de trabajo y refuerza la eficiencia operativa. De esta forma, el proyecto contribuye a la modernización de los sistemas de gestión empresarial y al fortalecimiento de la capacidad tecnológica de la organización.
\end{itemize}