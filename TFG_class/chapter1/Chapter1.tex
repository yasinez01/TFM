\chapter[Introducción]{Introducción nombre largo}
\label{chap:intro}

% Este capitulo no será usado en el Principal.tex, solo existe como documentación de etiquetas.

%intorduccion del trabajo a realizar
%motivacion
%introducir proyecto en el que trabajas
%intruducir la empresa




\section{Esto es una sección}
Aquí tenemos una imagen referenciada \ref{fig:ej1}. Una dirección \href{https://www.copernicus.eu/es}{Mi dirección}%\footnotemark. 

sdagyhadrhj\footnote{textofaas fasfa }
\begin{figure} [h!btp]
	\centering
	\includegraphics[width=0.7\textwidth]{fig/example-image.png}
	\caption[Nombre reducido para tabla de figuras]{Real caption\footnotemark}
	\label{fig:ej1}
\end{figure}
\footnotetext{https://www.copernicus.eu/es}

Para generar entradas en el índice de palabras\index{parabras}. SATA\index{SATA}.

\subsection{Esto es una subsección}
%un comentario
Una lista de parámetros: % con itemize se muestra una lista
\begin{itemize}
	\item uno.
	\item dos.
	\item tres.
\end{itemize}
Una lista enumerada
\begin{enumerate}
	\item  uno.
	\item  dos.
\end{enumerate}

\begin{description}
    \item[elemento:] Definicion
\end{description}


Vamos a citar \ldots \cite{lorenzi2011inpainting}
\begin{table}[H]\caption{Mi tabla de ejemplo}\label{tab:ej}
\begin{center}
	\begin{tabular}{c c c}
		Nombre & Medida & Otra cosa\\
		\hline
		10 & 10 & 4\\
	\end{tabular}
\end{center}
\end{table}


\begin{tabular}{|cc|c|c|c|}
\hline
fghaf &  &  &  &  \\

 & afgg &  &  &  \\
\hline
\hline
 &  &  &  &  \\
\hline
 &  &  &  &  \\
\hline
 &  &  &  &  \\
\hline
 &  & asgdf &  &  \\
\hline
 &  &  &  &  \\
\hline
 & gasd &  &  &  \\
\hline
 &  &  & agsd &  \\
\hline
 &  &  &  &  \\
\hline
 &  &  &  &  \\
\hline
\end{tabular}