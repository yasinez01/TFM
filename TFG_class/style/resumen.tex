\thispagestyle{empty}

\chapter*{Resumen}
Este Trabajo de Fin de Máster describe el diseño y desarrollo del Ecosistema de Aprendizaje, una aplicación web pensada para mejorar la gestión de la formación interna en Tracasa Global y Tracasa Instrumental. El proyecto surge a partir de la necesidad de contar con una plataforma única que permita organizar de forma más sencilla y ordenada todo lo relacionado con la formación de los empleados, desde que se solicita una acción formativa hasta su evaluación final.

La aplicación reúne en un solo entorno funciones clave como la solicitud de nuevas formaciones, su aprobación por parte de responsables y directores, y el seguimiento del estado de cada proceso. Además, ofrece acceso directo a plataformas externas de aprendizaje y ciberseguridad, ampliando así las opciones de desarrollo profesional disponibles. También incluye herramientas que mejoran el control y el seguimiento, como el registro de cambios, la gestión de asistentes y docentes, y la recogida de valoraciones y encuestas de eficacia.

Otro eje fundamental del proyecto es la gestión de los planes de formación anuales. La aplicación permite a los distintos departamentos proponer, revisar y combinar acciones formativas, así como mantener un control de las propuestas aceptadas y su relación con las formaciones creadas. Esta funcionalidad contribuye a una planificación más estratégica y coherente con las necesidades globales de la organización.

En definitiva, este trabajo representa un paso adelante en la digitalización de la formación corporativa y refuerza la apuesta de la empresa por la mejora continua, el aprendizaje constante y una gestión del talento más ágil y transparente.

\vspace{1cm}
Palabras clave: Ecosistema de Aprendizaje, aplicación web, formación corporativa, gestión de la formación, Tracasa Global, Tracasa Instrumental, digitalización, planes de formación.

\clearpage
\thispagestyle{empty}
\hfill
\clearpage