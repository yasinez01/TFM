\thispagestyle{empty}

\chapter*{Resumen}
Este Trabajo de Fin de Máster presenta el diseño y desarrollo del Ecosistema de Aprendizaje, una aplicación web creada para mejorar la gestión de la formación corporativa en Tracasa Global y Tracasa Instrumental. El proyecto responde a la necesidad de disponer de un entorno digital unificado que permita coordinar de forma eficiente todos los procesos relacionados con la formación de los empleados, desde la solicitud inicial hasta la valoración final de cada acción formativa.

La herramienta integra en un mismo espacio funciones esenciales como la solicitud de nuevas formaciones, la aprobación por parte de responsables y directores y el seguimiento del progreso de cada proceso. También facilita el acceso directo a plataformas externas de aprendizaje y de ciberseguridad, lo que amplía las oportunidades de desarrollo profesional disponibles para los empleados. Además, incorpora mecanismos que mejoran la trazabilidad, como el histórico de cambios, la gestión de asistentes y docentes y la recogida de valoraciones y encuestas de eficacia.

Otro eje fundamental del proyecto es la gestión de los planes de formación anuales. La aplicación permite a los distintos departamentos proponer, revisar y combinar acciones formativas, así como mantener un control centralizado de las propuestas aceptadas y su relación con las formaciones creadas. Esta funcionalidad contribuye a una planificación más estratégica y coherente con las necesidades globales de la organización.

En conjunto, este trabajo supone un avance significativo en la digitalización de la formación corporativa y refuerza el compromiso de la empresa con la mejora continua, el aprendizaje permanente y una gestión del talento más ágil y transparente.

\vspace{1cm}
Palabras clave: Ecosistema de Aprendizaje, aplicación web, formación corporativa, gestión de la formación, Tracasa Global, Tracasa Instrumental, digitalización, planes de formación.

\clearpage
\thispagestyle{empty}
\hfill
\clearpage