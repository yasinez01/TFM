\chapter[Proyecto]{Proyecto}
\label{Chap4}

La aplicación \textit{Ecosistema de Aprendizaje} es un entorno digital que nos va a permitir, desde la organización, gestionar el aprendizaje y el desarrollo profesional de una manera más ágil y dinámica. Se trata de un entorno que irá evolucionando, totalmente adaptado a las necesidades de la empresa. 
La aplicación nos ofrece varias funcionalidades y ciertas novedades:  

\begin{enumerate}
	\item Posibilidad de cada persona pueda solicitar formación, que será validada por el manager y PyT.  
	\item Posibilidad de conocer qué acciones formativas se van a hacer en la empresa y solicitar incluirse en ella. También será validado por el manager y PyT. 
	\item Acceder a una plataforma de aprendizaje donde cada persona podrá desarrollase en aquellas materias que puedan resultar de interés. Aquellas acciones que sean por interés personal y no estén relacionadas con el desempeño de su puesto, se recibirán fuera de jornada laboral.
\end{enumerate}

Cuando accedemos a la aplicación, visualizamos la pantalla HOME, donde podremos elegir la acción que vayamos a realizar:  

\begin{enumerate}
	\item Acceso a la plataforma de aprendizaje GoodHabitz.
	\item Acceso a la plataforma de Tracafit.
	\item Acceso a la plataforma de ciberseguridad.
	\item Solicitar una nueva acción formativa.
	\item Visualizar las acciones previo a su comienzo y solicitar unirse a ella. 
\end{enumerate}
\begin{figure} [h!btp]
	\centering
	\includegraphics[width=0.7\textwidth]{fig/home.png}
	\caption[Pantalla home con los accesos directos]{Pantalla home con los accesos directos\footnotemark}
	\label{fig:ej1}
\end{figure}

\section{Acceso a plataforma de aprendizaje}
Para acceder a la plataforma de aprendizaje debemos posicionarnos en la pantalla Home y seleccionar en GoodHabitz 
Al acceder a la plataforma, por defecto, y sólo la primera vez, aparecerá un mensaje para aceptar la política de privacidad y acuerdo de usuario. 
\begin{figure} [h!btp]
	\centering
	\includegraphics[width=0.7\textwidth]{fig/plataformaAprendizaje.png}
	\caption[Pantalla plataforma de aprendizaje]{Pantalla plataforma de aprendizaje\footnotemark}
	\label{fig:ej2}
\end{figure}
A continuación, encontraremos, en la parte superior de la pantalla, dos opciones a las que podremos acceder:  
\begin{itemize}
	\item Tus cursos: donde podremos encontrar aquellas acciones formativas que cada usuario haya realizado, esté realizando en ese momento o, también podemos encontrar aquellas formaciones que la empresa haya sugerido realizar
	\item Todos los cursos: podremos ver el catálogo de acciones formativas a las que podemos acceder y cursar.
\end{itemize}

\section{Acceso directo a la página de Tracafit}
Acceso directo al moodle de tracasa.

\section{Acceso directo a la plataforma de ciberseguridad}
Acceso directo a contenidos para mejorar las capacidades en ciberseguridad. 
\begin{figure} [h!btp]
	\centering
	\includegraphics[width=0.7\textwidth]{fig/plataformaCiberseguridad.png}
	\caption[Plataforma de ciberseguridad]{Plataforma de ciberseguridad\footnotemark}
	\label{fig:ej3}
\end{figure}

\section{Solicitar nueva acción formativa}
Se proporciona dos accesos para poder solicitar una formación. Podemos crear una formación extraordinaria o planificada. La diferencia que hay es que en el caso de la planificada debemos seleccionar una propuesta de plan de formación (que explicaremos más tarde) que asignaríamos a la formación que vamos a crear y para el caso de extraordinaria no se le asigna.

Al acceder a la opción de \textit{Solicitar formación extraordinaria} encontraremos la siguiente pantalla:  
\begin{figure} [h!btp]
	\centering
	\includegraphics[width=0.7\textwidth]{fig/cuestionarioIntake.png}
	\caption[Cuestionario INTAKE para solicitar formación]{Cuestionario INTAKE para solicitar formación\footnotemark}
	\label{fig:ej4}
\end{figure}
Un cuestionario de 10 preguntas para recoger información sobre la necesidad formativa u oportunidad de mejora. Este cuestionario solo aparece para el caso de formación extraordinaria.

Nos encontramos con otra pestaña \textit{Datos generales} que sí aparece para ambos tipos de formación.
que nos muestra la siguiente pantalla:
\begin{figure} [h!btp]
	\centering
	\includegraphics[width=0.7\textwidth]{fig/pestanaDatosGenerales.png}
	\caption[Formulario de datos generales a rellenar]{Formulario de datos generales a rellenar\footnotemark}
	\label{fig:ej5}
\end{figure}

Cumplimentar los datos de la segunda pantalla:
\begin{itemize}
	\item \textbf{Nombre:} Nombre de la acción formativa.
	\item \textbf{Área:} Área a la que pertenece el creador de la solicitud.
	\item \textbf{Aprobador/a:} Responsable directo del/los asistentes. En caso de asistir personal de varias áreas, identificar como aprobador al responsable del área de la que asistan más personas.
	\item \textbf{Validador/a:} Director de departamento al que pertenecen los asistentes. En caso de asistir personal de varios departamentos, identificar como aprobador al director del departamento del que asistan más personas.
	\item \textbf{Valorador/a:} Encargado/a de valorar la formación. Puede haber más de un/a valorador/a para cada formación. Por defecto se asigna como valorador/a el aprobador/a.
	\item \textbf{Planificación:} Ubicarla en el trimestre en el que se imparte la acción.
	\item \textbf{Modalidad de la formación:} Modalidad en la que se imparte (Online, Streaming, presencial, mixta…).
	\item \textbf{Tipo de formación:} Si la formación es interna (el docente es de plantilla), externa (el docente es una empresa/persona ajena a la empresa), mixta (se combinan ambas modalidades).
	\item \textbf{Medio pedagógico:} Comunidades en prácticas (aprendizaje en comunidad o social), aprendizaje experiencial (Learning by doing), aprendizaje tradicional (formación tradicional).
	\item \textbf{Tipo de contenido:} Si se trata de conocimientos técnicos, Soft Skills, Idiomas, Agile…
	\item \textbf{Lugar:} Dónde tiene lugar la acción formativa.
	\item \textbf{Tipo de coste:} Coste (si la acción implica un coste en la matriculación) o gratuita (no supone coste en la inscripción).
	\item \textbf{Código:} Aparece cumplimentada por defecto. No rellenable.
	\item \textbf{Unidades:} Número de personas para asistir.
	\item \textbf{Importe:} En caso de que suponga coste, indicar cuantía por persona sin IVA.
	\item \textbf{Total:} Cuantía total sin IVA. No rellenable.
	\item \textbf{Fecha Inicio:} Fecha en la que tiene lugar el inicio de la acción formativa.
	\item \textbf{Fecha Fin:} Fecha en la que finaliza la acción formativa.
	\item \textbf{Horas de formación:} Número de horas que dura la acción formativa.
	\item \textbf{Sala:} La sala donde se impartirá la acción formativa.
	\item \textbf{Días aviso valoración:} Aparece cumplimentada por defecto. No rellenable.
	\item \textbf{Tipo de plan de formación:} Planificado o Extraordinario. Según qué tipo de solicitud hemos solicitado tendrá un valor u otro, con la posibilidad de poder cambiarlo.
	\item \textbf{Plan de formación anual:} Plan para formar en el año actual. En el caso de solicitud extraordinaria este campo estará deshabilitado y, para el caso de planificada, tendrá como valor la propuesta del plan de formación seleccionada, también con la opción de poder cambiarla.
	\item \textbf{Mostrar en la home:} Opción para mostrar o no en la pestaña home.
	\item \textbf{Es bonificable:} Opción para indicar si es bonificable o no la acción formativa.
	\item \textbf{Contenido y/o programa:} Espacio destinado a alojar el link del programa de contenidos, en caso de disponer de él, o adjuntarlo en caso de tenerlo como documento.
	\item \textbf{Temario:} Espacio destinado a adjuntar el temario en caso de tenerlo como documento.
	\item \textbf{Observaciones:} Observaciones opcionales para indicar en la acción formativa.
\end{itemize}

Si hemos seleccionado la opción de \textit{Solicitar formación planificada} nos aparece 
\begin{figure} [h!btp]
	\centering
	\includegraphics[width=0.7\textwidth]{fig/seleccionarPropuestaFormacion.png}
	\caption[Ventana para seleccionar propuesta a asignar a la formación]{Ventana para seleccionar propuesta a asignar a la formación\footnotemark}
	\label{fig:ej6}
\end{figure}

Es un listado de propuestas aceptadas de los planes de formación que estén aprobados y activos. Se selecciona una propuesta y haciendo click en el botón \textit{Aceptar} nos carga solo la pestaña de \textit{Datos generales} con los campos \textit{Tipo de plan de formación} y \textit{Plan de formación} automáticamente rellenados.