\chapter[Conclusiones y Trabajo Futuro]{Conclusiones y Trabajo Futuro}
\label{Chap7}

Este proyecto ha dado lugar al diseño, desarrollo e implementación del \textit{Ecosistema de Aprendizaje} para Tracasa Global y Tracasa Instrumental, una plataforma web destinada a optimizar la gestión de la formación interna. La solución ha permitido centralizar todo el proceso, desde la solicitud de formaciones hasta su evaluación final, facilitando la organización, trazabilidad y transparencia de las acciones formativas.

A nivel funcional, la aplicación ha demostrado ser útil para la gestión de solicitudes, la planificación de formaciones y la recolección de valoraciones. La posibilidad de acceder a plataformas externas de aprendizaje, como GoodHabitz y herramientas de ciberseguridad, ha ampliado las opciones de desarrollo para los empleados, alineando las necesidades individuales con las prioridades organizativas.

Desde un punto de vista técnico, con las tecnologías que se han usado se ha conseguido un desarrollo robusto, escalable y seguro. La arquitectura por capas y el patrón \texttt{CQRS} han garantizado que el sistema sea fácil de mantener y expandir, facilitando futuras mejoras sin comprometer la estabilidad.

A nivel de resultados, el sistema ha tenido un impacto positivo en la agilidad de los procesos formativos, lo que se ha traducido en una mayor satisfacción de los empleados y en una gestión más eficiente del talento.

\section*{Trabajo Futuro}

Aunque el \textit{Ecosistema de Aprendizaje} ya ha cumplido con los objetivos iniciales, existen varias mejoras y nuevas funcionalidades que podrían potenciar aún más su impacto en el futuro:

\begin{itemize}
	\item \textbf{Integración con más plataformas de aprendizaje:} Expandir la plataforma con nuevas herramientas externas para ofrecer a los empleados una mayor variedad de contenidos formativos, lo que enriquecería aún más el ecosistema.
	
	\item \textbf{Análisis automatizado de resultados:} Incorporar inteligencia artificial y análisis de datos para procesar los resultados de las encuestas de eficacia formativa, lo que ayudaría a identificar tendencias y áreas de mejora de manera más eficiente.
	
	\item \textbf{Mejoras en la usabilidad:} A medida que se recojan más datos de los usuarios, sería ideal mejorar la interfaz de usuario para que sea aún más accesible, especialmente para aquellos empleados con menor experiencia tecnológica.
	
	\item \textbf{Personalización de la formación:} Desarrollar una funcionalidad que permita personalizar las formaciones en función de las necesidades específicas de cada empleado, basándose en su perfil y rendimiento en formaciones anteriores.
	
   \item \textbf{Ampliación de pruebas automatizadas:} Una tarea clave para el futuro será la implementación de más pruebas automatizadas para garantizar la estabilidad del sistema durante el ciclo de vida del proyecto. Esto permitirá detectar de manera temprana posibles errores y evitar impactos negativos en las versiones posteriores del sistema, mejorando así la fiabilidad del producto.

   \item \textbf{Mejoras en ciberseguridad:} Es esencial implementar una serie de medidas adicionales de seguridad, como la autenticación multifactor y el cifrado avanzado de datos. También se deberían realizar auditorías periódicas de seguridad para detectar vulnerabilidades y proteger la información sensible de los usuarios frente a posibles amenazas externas.
   
   \item \textbf{Desarrollo de tutoriales en video:} Creación de tutoriales en video que guíen a los usuarios a través de las funcionalidades clave de la aplicación. Estos videos facilitarían la formación y ayudarían a los empleados a familiarizarse con el sistema de manera más interactiva y visual. Además, proporcionarían una manera accesible de aprender cómo utilizar la plataforma sin necesidad de asistencia directa, mejorando la experiencia del usuario y reduciendo el tiempo necesario para adoptar completamente la herramienta.
   
\end{itemize}

En resumen, este proyecto ha sido un avance significativo para mejorar cómo se gestiona el aprendizaje y el desarrollo profesional en ambas empresas. Al crear el Ecosistema de Aprendizaje, hemos logrado simplificar y agilizar los procesos de formación, lo que no solo facilita la organización interna, sino que también impulsa el crecimiento de los empleados. Este es solo el comienzo, ya que las próximas actualizaciones seguirán enfocadas en reforzar la visión de Tracasa Global y Tracasa Instrumental de fomentar un entorno de trabajo flexible, innovador y siempre en evolución. El objetivo es continuar creando soluciones que apoyen a las personas y equipos a crecer y adaptarse en un mundo corporativo que cambia rápidamente.